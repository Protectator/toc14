% YOUR SOLUTION TO PROBLEM 2 GOES HERE
% THE LATEX CODE YOU PUT HERE WILL BE INLINED (IF YOU NEED ADD ANY PACKAGES DO IT IN THE MAIN TEMPLATE FILE)

The shortest path is easy to find in polynomial time using e.g. Dijstra's algorithm. Checking if this path is shorter than k is also done in polynomial time. So we can say that $SHORT\_PATH$ is in P.\\

$LONG\_PATH$ is in NP because we can check in polynomial time that it is a path and that its length is greater than or equal to k. If we take a special case of $LONG\_PATH$ that is an Hamiltonian Path where k is equal to the number of vertices minus 1, we can conclude that $LONG\_PATH$ is NP-complete as we know that finding an Hamiltonian path is NP-complete.