% YOUR SOLUTION TO PROBLEM 1 GOES HERE
% THE LATEX CODE YOU PUT HERE WILL BE INLINED (IF YOU NEED ADD ANY PACKAGES DO IT IN THE MAIN TEMPLATE FILE)

a) Let z be a word in $\{0, 1\}*$, and $z^R$ its reverse. $L_1$ must not include $zz^R$ since $(zz^R)^R = zz^R$. 
But, by the pigeonhole principle, there has to be a $z'$ != $z$ such that a DFA for $L_1$ cannot distinguish between $z$ and $z'$. 
This means that a DFA is unable of both rejecting all palindromes (including $zz^R$) and accepting all non-palindromes (including $z'z^R$), therefore $L_1$ must be irregular. \\

b) Again, by the pigeonhole principle, there has to exist a $k$ and a $k'$ such that a DFA recognizing $L_2$ cannot distinguish between $k$ 1s and $k'$ 1s. 
But if the $y$ that follows has $Y$ 1s where $k < Y < k'$, $k$ 1s followed by $y$ must be rejected while $k'$ 1s followed by $y$ must be accepted.
This is not possible, we cannot build such a DFA, therefore $L_2$ is irregular. \\

c) Same as... oh, wait, it's actually not the same thing, there's a trick!
We can freely move 1s from the prefix to the last part, and the last part can match a lot of things. $L_3$ is equivalent to $L_3' := \{ 1y' | y' contains at least one 1 \}$. Let's prove it.
$L_3'$ is trivially a subset of $L_3$ since it's equivalent to $L_3 with k = 1$.
$L_3$ is a subset of $L_3'$: the prefix in $L_3$ can be "split" into two parts: one 1, and zero or more 1s. The first part corresponds to the prefix in $L_3'$, the second part is absorbed into $y'$.
Since $y$ contains at least $k$ 1s and $k >= 1$, the condition for $y'$ (at least one 1) is met. Thus, surprisingly, $L_3$ is a subset of $L_3'$.
Since $L_3$ is a subset of $L_3'$ and vice-versa, $L_3 = L_3'$.
$L_3'$ - in English, the set of words beginning by 1 and containing another 1 somewhere - is regular, here's a DFA that matches it:

\begin{figure}[H]
\centering
\begin{tikzpicture}[shorten >=2pt, auto, node distance=4 cm, scale = 1, transform shape]
    
\node[state, initial](q0) {$q0$};
\node[state] (q1) [right of=q0] {$q1$};
\node[state, accepting] (q2) [right of=q1] {$q2$};
\node[state] (q3) [below of=q1] {$q3$};

\path[->] (q0) edge node {0} (q3);
\path[->] (q0) edge node [align=center] {1} (q1);

\path[->] (q1) edge [loop above] node {0} (q1);
\path[->] (q1) edge node [align=center] {1} (q2);

\path[->] (q2) edge [loop above] node {0, 1} (q2);

\path[->] (q3) edge [loop right] node {0, 1} (q3);

\end{tikzpicture}
\caption{Problem 1c DFA}
\end{figure}
