% YOUR SOLUTION TO PROBLEM 1 GOES HERE
% THE LATEX CODE YOU PUT HERE WILL BE INLINED (IF YOU NEED ADD ANY PACKAGES DO IT IN THE MAIN TEMPLATE FILE)

a) Let's start by assuming that $x_1$ = $y^r$. Therefore we know that by the pigeonhole principle, there has to be an $x_2$ != $y^r$ with $x_2$ != $x_1$ as there is a finite number of states for an infinite number of combinations. Therefore, we have somewhere collisions, which makes our DFA unable of both rejecting palindromes and accepting all non-palindromes.\\

b) By the pigeonhole principle, there has to exist a k' such that a DFA is in the same state after k 1's and k' 1's but then one of them can be accepted because the y that follows has A 1's where $k < A < k'$. That means that one of them has to be refused while the other has to be accepted, which is not possible with a DFA. \\

c) For this one, we assume that $x_1$ has enough 1's and $x_2$ hasn't. As this need to be true for every $k > 1$ we can assume that there is a k for which we will find a collision of the two x's. These would therefore lead to the same state. We can therefore not build a DFA for this one either