% YOUR SOLUTION TO PROBLEM 2 GOES HERE
% THE LATEX CODE YOU PUT HERE WILL BE INLINED (IF YOU NEED ADD ANY PACKAGES DO IT IN THE MAIN TEMPLATE FILE)

a) The start state of our DFA is a state which can go to itself on any input, or go on a 0 to a second state in a bloc of K states with each state going to the next one on any input (0 or 1), and the last one is the accepting state and has no transitions.

\begin{figure}[H]
\centering
\begin{tikzpicture}[shorten >=2pt, auto, node distance=4 cm, scale = 1, transform shape]

\node[state, initial](q1) {$q_1$};
\node[state](k) [right of=q1] {$K states$};
\node[state, accepting] (q2) [right of=k] {$q_3$};

\path[->] (q1) edge [bend left] node [align=center] {0} (k);
\path[->] (q1) edge [loop above] node [align=center] {0,1} (q1);


\path[->] (k) edge [bend left] node [align=center] {0,1} (q2);

\end{tikzpicture}
\caption{DFA $A$}
\end{figure}

b) Let us consider the set K of all words of length $k+1$ in ${0,1}*$. We want to prove that all of these words are pairwise distinguishable by L. Let $w_1$, $w_2$ be two different words in $K$. Since they are different, there has to be at least one bit that is not the same between them, which also implies that one of these words will have a $0$ where the other has a $1$. Let's say $w_1$ is the one with a $0$ and $w_2$ the one with a $1$, and the last change bit is named $B$. We can construct a suffix $u$ of length $X$ (whether its bits are 0 or 1 doesn't matter), where $X$ is $k - <number of bits after B in w_1>$ (thus $0 <= X <= k$).
This means that $w_1u$ will be accepted while $w_2u$ will be refused, therefore $w_1$ and $w_2$ are distinguishable by L. We juste proved that there exists a set $K$ of $2^{(k+1)}$ words which are pairwise distinguishable by L. By Theorem 1, we conclude that any DFA that implements L has to have $>= 2^{(k+1)}$ states.