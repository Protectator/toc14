\documentclass[a4paper,11pt]{article}
%
% HOW TO USE THIS TEMPLATE:
%
% MODIFY THE VALUE BELOW TO MAKE IT REFLECT THE CURRENT PROBLEM SET NUMBER
% (FOR EXAMPLE, FOR PROBLEM SET XX, THE LINE BELOW SHOULD BE: \def\mynumber{XX})
  \def\mynumber{6}
% 
% MODIFY THE VALUE BELOW TO MAKE IT REFLECT THE NUMBER OF PROBLEMS IN THE CURRENT PROBLEM SET
% (FOR EXAMPLE, IF THE CURRENT PROBLEM SET CONTAINS YY PROBLEMS, THE LINE BELOW SHOULD BE: \def\myproblemsnbr{YY})
  \def\myproblemsnbr{2}
%
%
%
%
%   ENTER THE LAST NAMES (AND ONLY LAST NAMES) OF ALL THE TEAM MEMBERS BELOWS
  \def\myname{K\"ung, Pirelli, Schubert, Dousse, Vu}
% REPLACE Doe, Smith, and Wilson WITH YOUR LAST NAMES
% NOTE: IF YOUR LAST NAME HAS ACCENTS, SOME LaTeX COMMANDS ARE:
%       for diaeresis/umlaut ", as in German, use \"a, \"u, \"o, etc.
%       for grave accent `, use \`e, \`a, \`u
%       for acute accent ', use \'a, \'e, \'u
%       for tilde ~, use \tilde{a}, \tilde{o}
%       for caret ^, use \hat{e}, \hat{o}
%	    if you need an accent on an i, say an umlaut, use \"\i{}
%		the \i{} is an i without its dot; e.g., na\"\i{}ve
%	    these all work well with lower-case letters,
%	        but only so-so with upper-case letters
%

% FOR EACH PROBLEM IN THE SET, WRITE YOUR SOLUTION IN A FILE
%   NAMED "problem1.tex", "problem2.tex", etc.
%   THESE SOLUTIONS FILES HAVE NO PREAMBLE AND
%   NO \begin{document} \end{document}
%   (THEY JUST GET INLINED WHEN RUNNING LaTeX)
%
% PLACE THESE SOLUTIONS FILES AND THE TEMPLATE IN THE SAME DIRECTORY
%
% RUN LaTeX
%
% IF ADDITIONAL PACKAGES ARE NEEDED, UNCOMMENT LINE BELOW
% AND ENTER THE PACKAGE NAMES
\usepackage{tikz}
\usepackage{booktabs}
\usepackage{float}
\usepackage{ amssymb }
\usepackage{amsmath}
\usepackage{array}
\usetikzlibrary{automata, arrows}


% DO NOT MODIFY ANYTHING BELOW THIS LINE
%%%%%

 \newcounter{problems}

  \setcounter{problems}{\myproblemsnbr}

  \usepackage{times,amsmath,amssymb,pslatex,graphicx,ifthen}
  \newcounter{problem}
  \setcounter{problem}{0}
  \makeatletter
  \def\ps@headings{%
    \let\@mkboth\markboth
    \def\@evenfoot{\hfil\large\sf Set \mynumber, Problem \theproblem\hfil}
    \def\@oddfoot{\@evenfoot}
    \def\@evenhead{\large\sc\myname\hfil\it\today\hfil\sf P.~\thepage}
    \def\@oddhead{\@evenhead}}
  \makeatother
  \pagestyle{headings}
  \advance\textwidth by16mm
  \advance\oddsidemargin by-8mm
  \advance\textheight by10mm
%%%%%

\begin{document}

\ifthenelse{\equal{\theproblems}{2}}{%
\begin{center}
  \LARGE\sf
  \begin{tabular}{||c|c||}
    \hline\hline
    Prob.~1 & Prob.~2 \\
    \hline
    &\\
    \hline\hline
  \end{tabular}
\end{center}}{}
\ifthenelse{\equal{\theproblems}{3}}{%
\begin{center}
  \LARGE\sf
  \begin{tabular}{||c|c|c||}
    \hline\hline
    Prob.~1 & Prob.~2 & Prob.~3 \\
    \hline
    &&\\
    \hline\hline
  \end{tabular}
\end{center}}{}
\ifthenelse{\equal{\theproblems}{4}}{%
\begin{center}
  \LARGE\sf
  \begin{tabular}{||c|c|c|c||}
    \hline\hline
    Prob.~1 & Prob.~2 & Prob.~3 & Prob.~4\\
    \hline
    &&&\\
    \hline\hline
  \end{tabular}
\end{center}}{}
\ifthenelse{\equal{\theproblems}{5}}{%
\begin{center}
  \LARGE\sf
  \begin{tabular}{||c|c|c|c|c||}
    \hline\hline
    Prob.~1 & Prob.~2 & Prob.~3 & Prob.~4 & Prob.~5\\
    \hline
    &&&&\\
    \hline\hline
  \end{tabular}
\end{center}}{}
\ifthenelse{\equal{\theproblems}{6}}{%
\begin{center}
  \LARGE\sf
  \begin{tabular}{||c|c|c|c|c|c||}
    \hline\hline
    Prob.~1 & Prob.~2 & Prob.~3 & Prob.~4 & Prob.~5 & Prob.~6\\
    \hline
    &&&&&\\
    \hline\hline
  \end{tabular}
\end{center}}{}
\ifthenelse{\equal{\theproblems}{7}}{%
\begin{center}
  \LARGE\sf
  \begin{tabular}{||c|c|c|c|c|c|c||}
    \hline\hline
    Prob.~1 & Prob.~2 & Prob.~3 & Prob.~4 & Prob.~5 & Prob.~6 & Prob.~7\\
    \hline
    &&&&&&\\
    \hline\hline
  \end{tabular}
\end{center}}{}
\ifthenelse{\equal{\theproblems}{8}}{%
\begin{center}
  \LARGE\sf
  \begin{tabular}{||c|c|c|c|c|c|c|c||}
    \hline\hline
    Prob.~1 & Prob.~2 & Prob.~3 & Prob.~4 & Prob.~5 & Prob.~6 & Prob.~7 & Prob.~8\\
    \hline
    &&&&&&&\\
    \hline\hline
  \end{tabular}
\end{center}}{}

\bigskip

\begin{flushleft}
{\Large Team members: \myname}
\end{flushleft}


\bigskip


%%%%%%%%%%%%%%%%%%%%%%%%%%%%%%%%%%%%%%%%%%%%%%%%%%%%%%%%%%%%
\begin{flushleft}
  \addtocounter{problem}{1}
  \large\sf Problem \theproblem .
\end{flushleft}

% YOUR SOLUTION TO PROBLEM 1 GOES HERE
% THE LATEX CODE YOU PUT HERE WILL BE INLINED (IF YOU NEED ADD ANY PACKAGES DO IT IN THE MAIN TEMPLATE FILE)
This is a simple DFA. The $E$ states are reached with an even number of 0s while the $O$ states are reached with an odd number of 0s. Only the $E$ states are accepting.
The first 1 in the input will switch from $E_1/O_1$ to $E_2/O_2$, and the second one will go to $F$, which is the failure state since more than one 1 has been detected. \\

\begin{figure}[H]
\centering
\begin{tikzpicture}[shorten >=2pt, auto, node distance=4 cm, scale = 1, transform shape]
    
\node[state, initial, accepting](q1) {$E_1$};
\node[state, accepting] (q2) [right of=q1] {$E_2$};
\node[state] (q3) [right of=q2] {$F$};
\node[state] (q4) [below of=q1] {$O_1$};
\node[state] (q5) [below of=q2] {$O_2$};

\path[->] (q1) edge [bend right] node {0} (q4);
\path[->] (q1) edge node [align=center] {1} (q2);

\path[->] (q2) edge [bend right] node [align=left] {0} (q5);
\path[->] (q2) edge  node [align=center] {1} (q3);


\path[->] (q4) edge [bend right] node [align=center] {0} (q1);
\path[->] (q4) edge node [align=center] {1} (q5);

\path[->] (q3) edge [loop right] node [align=center] {0, 1} (q2);

\path[->] (q5) edge [bend right] node [align=center] {0} (q2);
\path[->] (q5) edge node [align=center] {1} (q3);

\end{tikzpicture}
\caption{Problem 1 DFA}
\end{figure}
    

\addtocounter{problems}{-1}

%%%%%%%%%%%%%%%%%%%%%%%%%%%%%%%%%%%%%%%%%%%%%%%%%%%%%%%%%%%%
\newpage
\begin{flushleft}
  \addtocounter{problem}{1}
  \large\sf Problem \theproblem .
\end{flushleft}

% YOUR SOLUTION TO PROBLEM 2 GOES HERE
% THE LATEX CODE YOU PUT HERE WILL BE INLINED (IF YOU NEED ADD ANY PACKAGES DO IT IN THE MAIN TEMPLATE FILE)

Let L be a language in P.
Let A be a polynomial-time algorithm that decides L.
We want to create $A'$, a polynomial type algorithm that decides $L^*$, given an input w consisting of characters $c_1c_2...c_n$.\\
To do so, using dynamic programming, we build a matrix M such that $M[i,j]$ is true iff $c_i...c_j \in L^*$. 
This matrix is built with a bottom-up approach, where $M[i,j]$ can be true if $c_i...c_j \in L$, or if $c_i...c_j$ can be divided in segments $s \in L\ \forall\ s$ (and therefore $c_i...c_j \in L^*$). \\
The process of building that matrix is done in polynomial time, more precisely in $O(a^3)$ where $a$ is the complexity of A since, for each pair $(i,j) \in (1..n, 1..n)$, there is one execution of A to check if $c_i...c_j\in L$ and $(j - i)$ executions of A to check if $c_i...c_j$ is made up of two parts $p \in L\ \forall\ p$. \\
Once this matrix is built, $w \in L^*\ iff\ M[1,n]$ is true.
\addtocounter{problems}{-1}

%%%%%%%%%%%%%%%%%%%%%%%%%%%%%%%%%%%%%%%%%%%%%%%%%%%%%%%%%%%%
\ifthenelse{\equal{\theproblems}{0}}
{}%
{%
\newpage
\begin{flushleft}
  \addtocounter{problem}{1}
  \large\sf Problem \theproblem .
\end{flushleft}

% YOUR SOLUTION TO PROBLEM 3 (IF THERE IS ONE) GOES HERE
% THE LATEX CODE YOU PUT HERE WILL BE INLINED (IF YOU NEED ADD ANY PACKAGES DO IT IN THE MAIN TEMPLATE FILE)

We want to prove the following:

\begin{align*}
\frac {L}{2} :=& \{x \in \Sigma * | \exists y : yx \in L, |x| = |y| \}
\end{align*}

We haven proven in problem 2 that the inverse of a regular language is regular. Let's assume we have some way of proving that the first half of a language is regular; we could prove that the first half of a reversed language is regular, which would also mean that the reverse of the first half of a reversed language is regular. A bit tricky, but it does correspond to the second half.

Let us prove our assumption; that the first half of a regular language is regular.


Let $L$ be a regular language. Let $half(L) := \{ x | \exists y : xy \in L, |x| = |y| \}$. We want to prove that $half(L)$ is regular.

Let $M < Q, \Sigma, \delta, q_0, F>$ be a DFA that accepts $L$, which must exist because $L$ is regular. Supposing that $\hat \delta (q_0, x) = q_i$, i.e. the input $x$ leads $M$ to the state $q_i$, we must check if $\exists y : |y| = |x|$ that satisfies $\hat \delta (q_i, y) \in F$. 

Let $S_n \subset Q$ be the set of states that lead to an accepting state for some input (not necessarily any input) of length n. 
It is clear that for $x$ of length $n$ and $\hat \delta(q_0, x) = q_i$, $q_i \in S_n \rightarrow x \in \frac L 2$.  $(*)$
$S_0 = F$, since by definition of $S_n$ there are no more steps needed. $S_{n+1}$ can be easily computed from $S_n$ and $\delta$; it is the set of states that can transition to a state in $S_n$.
We need a DFA that will keep track of $S_n$ as we go through $L$. Let $M'$ be a new DFA whose states are in $(Q, Q*)$ i.e. they're pairs of one state in $Q$ and a set of states in $Q$. The transition function $\delta'$ of $M'$ takes an input $x$ of length $n$ and yields $(\hat \delta(q_0, x), S_n)$.
The start state of $M'$ is $(q_0, F)$, and the accepting states are $(q, S) \in (Q, Q^*)$ where $q \in S$; in common English, "the states reached after n inputs that can also reach an accepting state after n inputs". 
$(*)$ above allows us to show that $M'$ indeed accepts $half(L)$, and therefore $half(L)$ is regular.

(source/inspiration for the $half(L)$ proof: \url{http://www-bcf.usc.edu/~breichar/teaching/2011cs360/half\%28L\%29example.pdf})

\addtocounter{problems}{-1}
}
%%%%%%%%%%%%%%%%%%%%%%%%%%%%%%%%%%%%%%%%%%%%%%%%%%%%%%%%%%%%
\ifthenelse{\equal{\theproblems}{0}}
{}%
{%
\newpage
\begin{flushleft}
  \addtocounter{problem}{1}
  \large\sf Problem \theproblem .
\end{flushleft}

\input{problem4.tex}
\addtocounter{problems}{-1}
}
%%%%%%%%%%%%%%%%%%%%%%%%%%%%%%%%%%%%%%%%%%%%%%%%%%%%%%%%%%%%%%%%%%
\ifthenelse{\equal{\theproblems}{0}}
{}%
{%
\newpage
\begin{flushleft}
  \addtocounter{problem}{1}
  \large\sf Problem \theproblem .
\end{flushleft}

\input{problem5.tex}
\addtocounter{problems}{-1}
}
%%%%%%%%%%%%%%%%%%%%%%%%%%%%%%%%%%%%%%%%%%%%%%%%%%%%%%%%%%%%
\ifthenelse{\equal{\theproblems}{0}}
{}%
{%
\newpage
\begin{flushleft}
  \addtocounter{problem}{1}
  \large\sf Problem \theproblem .
\end{flushleft}

\input{problem6.tex}
\addtocounter{problems}{-1}
}
%%%%%%%%%%%%%%%%%%%%%%%%%%%%%%%%%%%%%%%%%%%%%%%%%%%%%%%%%%%%
\ifthenelse{\equal{\theproblems}{0}}
{}%
{%
\newpage
\begin{flushleft}
  \addtocounter{problem}{1}
  \large\sf Problem \theproblem .
\end{flushleft}

\input{problem7.tex}
\addtocounter{problems}{-1}
}
%%%%%%%%%%%%%%%%%%%%%%%%%%%%%%%%%%%%%%%%%%%%%%%%%%%%%%%%%%%%
\ifthenelse{\equal{\theproblems}{0}}
{}%
{%
\newpage
\begin{flushleft}
  \addtocounter{problem}{1}
  \large\sf Problem \theproblem .
\end{flushleft}

\input{problem8.tex}
\addtocounter{problems}{-1}
}
\end{document}
